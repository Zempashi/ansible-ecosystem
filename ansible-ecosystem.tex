\PassOptionsToPackage{dvipsnames}{xcolor}
\documentclass{beamer}
\usepackage[utf8]{inputenc}
\usepackage[english]{babel}
%\usepackage{verbatim}
\usepackage{graphicx}
\usepackage{amsmath}
\usepackage{array}
\usepackage{tikz}
\usetikzlibrary{arrows,calc,intersections,through,decorations.pathmorphing,backgrounds,positioning,fit}
\usepackage[T1]{fontenc}
\usepackage{xcolor}
\usepackage{transparent}
\usepackage{fancyvrb}
\usepackage{listings}
\usepackage{minted}
\usepackage{ulem}
\usepackage{hyperref}
\usetheme{Madrid}

\AtBeginSection[]
{
  \begin{frame}<beamer>
      \frametitle{Syllabus}
      {\scriptsize\tableofcontents[currentsection,hideothersubsections]
      }
    \end{frame}
}

%\fvset{commandchars=\\\{\}}

\title{Ansible ecosystem}
\author{Julien Girardin}
\date{\today}

\titlegraphic{
    \includegraphics[width=0.06\textwidth]{images/ansible_badge.png}
    \hspace{0.1cm}
    \includegraphics[width=0.3\textwidth]{images/ansible.png}
}

\begin{document}

\maketitle{}

%\begin{frame}
%    \begin{exampleblock}{Versions disclaimer}
%        Should work on ansible >= 2.0
%
%        (Special mentions if not applicable)
%    \end{exampleblock}
%\end{frame}

\section{Ansible package}

\begin{frame}
    \begin{itemize}
        \item {\bf ansible} Run a single command on distant hosts. \\
            Usefull tasks: \textit{ping, command, service, file, copy}
        \item {\bf ansible-playbook} Run playbook file on distant hosts \\
             The real power of ansible !
        \item {\bf ansible-pull} Run a playbook against the local machine \\
        \item {\bf ansible-galaxy} Install roles from remote location. \\
            sources: \textit{github, http, tar}
        \item {\bf ansible-vault} Encrypt source file.
            Commit password encrypted by a symmetric master key
        \item {\bf \color{gray} ansible-container \textit{- beta -}} Build docker container
    \end{itemize}
\end{frame}

\subsection{ansible cli}

\begin{frame}[fragile]
    \frametitle{ansible cli}
    \footnotesize{
        \begin{minted}{shell}
$ export ANSIBLE_HOSTS=/etc/ansible/ec2.py
        \end{minted}
        \begin{minted}{shell}
$ export EC2_INI_PATH=/etc/ansible/ec2.ini
        \end{minted}
        \begin{minted}{shell}
$ ansible -m ping tag_ansible_slaves
10.1.2.193 | success >> {
    "changed": false,
    "ping": "pong"
}
10.1.2.136 | success >> {
    "changed": false,
    "ping": "pong"
}
10.1.2.137 | success >> {
    "changed": false,
    "ping": "pong"
}
        \end{minted}
    }
\end{frame}

\subsection{ansible-pull}

\begin{frame}[fragile]
    \frametitle{ansible-pull}
    \tiny{
        \inputminted{shell}{sources/ansible_pull.yml}
    }
\end{frame}

\subsection{ansible-galaxy}

\begin{frame}[fragile]
    \frametitle{ansible-galaxy}
    \begin{columns}
    \column{0.4\textwidth}
        \begin{block}{without galaxy}
        \footnotesize{
            \begin{Verbatim}[tabsize=4]
site.yml
subsite1.yml
subsite2.yml
groups_vars/
    prod/
        system.yml
        password.yml
    test/
        system.yml
        password.yml
host_vars/
roles/
    roles1/
    roles2/
    roles3/
    roles4/
        \end{Verbatim}
        }
         \end{block}
    \column{0.1\textwidth}
    \centering{VS}
    \column{0.4\textwidth}
        \begin{block}{with galaxy}
        \footnotesize{
            \begin{Verbatim}[tabsize=4]
site.yml
subsite1.yml
subsite2.yml
groups_vars/
    prod/
        system.yml
        password.yml
    test/
        system.yml
        password.yml
host_vars/
requirement.yml
roles/
            \end{Verbatim}
        }
    \end{block}
    \begin{minted}[breaklines]{shell}
$ ansible-galaxy install -r requirements.yml -p roles
    \end{minted}
    \end{columns}
\end{frame}

\subsection{ansible-vault}

\begin{frame}[fragile]
    \frametitle{ansible-vault}
    \begin{minted}{shell}
$ ansible-vault create group_vars/prod/password.yml
Vault password:
# $EDITOR open
$ git add group_vars/prod/password.yml
$ git commit -m 'Add prod password into ansible-vault'

$ ansible-playbook site.yml --ask-vault-pass
Vault password:
    \end{minted}
\end{frame}

\section{git}

\begin{frame}

\end{frame}


\section{jenkins}

\begin{frame}
    \frametitle{jenkins}
    \footnotesize{
    \begin{itemize}
        \item \href{https://wiki.jenkins-ci.org/display/JENKINS/Ansible+Plugin}
                    {https://wiki.jenkins-ci.org/display/JENKINS/Ansible+Plugin}
        \item \href{https://wiki.jenkins-ci.org/display/JENKINS/SSH+Agent+Plugin}
                    {https://wiki.jenkins-ci.org/display/JENKINS/SSH+Agent+Plugin}
        \item \href{https://wiki.jenkins-ci.org/display/JENKINS/Promoted+Builds+Plugin}
                    {https://wiki.jenkins-ci.org/display/JENKINS/Promoted+Builds+Plugin}
    \end{itemize}
    }
\end{frame}

\end{document}

% vim: set tabstop=2:
